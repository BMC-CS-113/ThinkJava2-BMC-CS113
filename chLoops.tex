\chapter{Loops}

Computers are often used to automate repetitive tasks, such as searching for text in documents.
Repeating tasks without making errors is something that computers do well and people do poorly.

Instead of using recursive methods (methods that call themselves), in this chapter
you will learn an iterative approach. 
You'll learn how to use \java{while} and \java{for} loops to add repetition to your code.

We have seen methods, like \java{countdown} and \java{factorial}, that use recursion to iterate.
Although recursion is elegant and powerful, it takes some getting used to (but hopefully you've 
gotten used to it now).
Java provides language features that make iteration much easier: the \java{while} and \java{for} statements.

\section{The while Statement}

\index{while}
\index{loop!while}
\index{statement!while}

Using a \java{while} statement, we can repeat the same code multiple times:

\begin{code}
int n = 3;
while (n > 0) {
    System.out.println(n);
    n = n - 1;
}
System.out.println("Blastoff!");
\end{code}

Reading the code in English sounds like this: ``Start with \java{n} set to 3.
While \java{n} is greater than 0, print the value of \java{n}, and reduce the value of \java{n} by 1.
When you get to 0, print Blastoff!''

The output is shown here:

\begin{stdout}
3
2
1
Blastoff!
\end{stdout}

The flow of execution for a \java{while} statement is as follows:

\begin{enumerate}

\item Evaluate the condition in parentheses, yielding \java{true} or \java{false}.

\item If the condition is \java{false}, skip the following statements in braces.

\item If the condition is \java{true}, execute the statements and go back to step 1.

\end{enumerate}

\index{loop}

This type of flow is called a {\bf loop}, because the last step ``loops back around'' to the first.
Figure~\ref{fig.while} shows this idea using a flowchart.

\begin{figure}[!ht]
\begin{center}
\includegraphics[width=190pt]{figs/while.pdf}
\caption{Flow of execution for a \java{while} loop.}
\label{fig.while}
\end{center}
\end{figure}

\index{loop body}
\index{infinite loop}
\index{loop!infinite}

The {\bf body} of the loop should change the value of one or more variables so that, eventually, the condition becomes \java{false} and the loop terminates.
Otherwise, the loop will repeat forever, which is called an {\bf infinite loop}:

\begin{code}
int n = 3;
while (n > 0) {
    System.out.println(n);
    // n never changes
}
\end{code}

This example will print the number \java{3} forever, or at least until you terminate the program.
An endless source of amusement for computer scientists is the observation that the directions on shampoo, ``Lather, rinse, repeat,'' are an infinite loop.

In the first example, we can prove that the loop terminates when \java{n} is positive.
But in general, it is not so easy to tell whether a loop terminates.
For example, this loop continues until \java{n} is 1 (which makes the condition \java{false}):

\begin{code}
while (n != 1) {
    System.out.println(n);
    if (n % 2 == 0) {         // n is even
        n = n / 2;
    } else {                  // n is odd
        n = 3 * n + 1;
    }
}
\end{code}

Each time through the loop, the program displays the value of \java{n} and then checks whether it is even or odd.
If it is even, the value of \java{n} is divided by 2.
If it is odd, the value is replaced by $3n+1$.
For example, if the starting value is 3, the resulting sequence is 3, 10, 5, 16, 8, 4, 2, 1.

Since \java{n} sometimes increases and sometimes decreases, there is no obvious proof that \java{n} will ever reach 1 and that the program will ever terminate.
For some values of \java{n}, such as the powers of two, we can prove that it terminates.
The previous example ends with such a sequence, starting when \java{n} is 16 (or $2^4$).

The hard question is whether this program terminates for {\em all} values of \java{n}.
So far, no one has been able to prove it {\em or} disprove it!
For more information, see \url{https://en.wikipedia.org/wiki/Collatz_conjecture}.
%The field of computer science is interested in these types of questions, because their answers give insight to the limits of what computers can and cannot do.


\section{Increment and Decrement}

Here is another \java{while} loop example; this one displays the numbers 1 to 5:

\begin{code}
int i = 1;
while (i <= 5) {
    System.out.println(i);
    i++;  // add 1 to i
}
\end{code}

\index{increment}
\index{decrement}

Assignments like \java{i = i + 1} don't often appear in loops, because Java provides a more concise way to add and subtract by one.
Specifically, \java{++} is the {\bf increment} operator; it has the same effect as \java{i = i + 1}.
And \java{--} is the {\bf decrement} operator; it has the same effect as \java{i = i - 1}.

%So far in this book we have only used (\java{=}) to assign values to variables.
%For convenience, Java provides other assignment operators that increase or decrease the value of a variable.

If you want to increment or decrement a variable by an amount other than \java{1}, you can use \java{+=} and \java{-=}.
For example, \java{i += 2} increments \java{i} by \java{2}:

\begin{code}
int i = 2;
while (i <= 8) {
    System.out.print(i + ", ");
    i += 2;  // add 2 to i
}
System.out.println("Who do we appreciate?");
\end{code}

And the output is as follows:

\begin{stdout}
2, 4, 6, 8, Who do we appreciate?
\end{stdout}


\section{The for Statement}

\index{for}
\index{loop!for}
\index{statement!for}

The loops we have written so far have three parts in common.
They start by initializing a variable, they have a condition that depends on that variable, and they do something inside the loop to update that variable.

\index{iteration}

Running the same code multiple times is called {\bf iteration}.
Methods that use iteration are called {\bf iterative}.
It's so common that there is another statement, the \java{for} loop, that expresses it more concisely.
For example, we can rewrite the 2-4-6-8 loop this way:

\begin{code}
for (int i = 2; i <= 8; i += 2) {
    System.out.print(i + ", ");
}
System.out.println("Who do we appreciate?");
\end{code}

\java{for} loops have three components in parentheses, separated by semicolons: the initializer, the condition, and the update:

\begin{enumerate}

\item The {\em initializer} runs once at the very beginning of the loop.
It is equivalent to the line before the \java{while} statement.

\item The {\em condition} is checked each time through the loop.
If it is \java{false}, the loop ends.
Otherwise, the body of the loop is executed (again).

\item At the end of each iteration, the {\em update} runs, and we go back to step~2.

\end{enumerate}

The \java{for} loop is often easier to read because it puts all the loop-related statements at the top of the loop.
Doing so allows you to focus on the statements inside the loop body.
Figure~\ref{fig.for} illustrates \java{for} loops with a flowchart.

%TODO: Add an initialization box to the flowchart

\begin{figure}[!ht]
\begin{center}
\includegraphics[width=190pt]{figs/for.pdf}
\caption{Flow of execution for a \java{for} loop.}
\label{fig.for}
\end{center}
\end{figure}

There is another difference between \java{for} loops and \java{while} loops: if you declare a variable in the initializer, it exists only {\em inside} the \java{for} loop.
For example:

\begin{code}
for (int n = 3; n > 0; n--) {
    System.out.println(n);
}
System.out.println("n is now " + n);  // compiler error
\end{code}

The last line tries to display \java{n} (for no reason other than demonstration), but it won't work.
If you need to use a loop variable outside the loop, you have to declare it {\em outside} the loop, like this:

\begin{code}
int n;
for (n = 3; n > 0; n--) {
    System.out.println(n);
}
System.out.println("n is now " + n);
\end{code}

Notice that the \java{for} statement does not say \java{int n = 3}.
Rather, it simply initializes the existing variable \java{n}.


\section{Nested Loops}
\label{nested}

\index{loop!nested}
\index{nested!loops}

Like conditional statements, loops can be nested one inside the other.
Nested loops allow you to iterate over two variables.
For example, we can generate a ``multiplication table'' like this:

\begin{code}
for (int x = 1; x <= 10; x++) {
    for (int y = 1; y <= 10; y++) {
        System.out.printf("%4d", x * y);
    }
    System.out.println();
}
\end{code}

\index{loop variable}
\index{variable!loop}
\index{inner loop}
\index{outer loop}

Variables like \java{x} and \java{y} are called {\bf loop variables}, because they control the execution of a loop.
In this example, the first loop (\java{for x}) is known as the ``outer loop'', and the second loop (\java{for y}) is known as the ``inner loop''.

Each loop repeats its corresponding statements 10 times.
The outer loop iterates from 1 to 10 only once, but the inner loop iterates from 1 to 10 each of those 10 times.
As a result, the \java{printf} method is invoked 100 times.

\index{format specifier}

The format specifier \java{\%4d} displays the value of \java{x * y} padded with spaces so it's four characters wide.
Doing so causes the output to align vertically, regardless of how many digits the numbers have:

\begin{stdout}
   1   2   3   4   5   6   7   8   9  10
   2   4   6   8  10  12  14  16  18  20
   3   6   9  12  15  18  21  24  27  30
   4   8  12  16  20  24  28  32  36  40
   5  10  15  20  25  30  35  40  45  50
   6  12  18  24  30  36  42  48  54  60
   7  14  21  28  35  42  49  56  63  70
   8  16  24  32  40  48  56  64  72  80
   9  18  27  36  45  54  63  72  81  90
  10  20  30  40  50  60  70  80  90 100
\end{stdout}

It's important to realize that the output is displayed row by row.
The inner loop displays a single row of output, followed by a newline.
The outer loop iterates over the rows themselves.
Another way to read nested loops, like the ones in this example, is: ``For each row \java{x}, and for each column \java{y}, \ldots''


\section{Characters}

Some of the most interesting problems in computer science involve searching and manipulating text.
In the next few sections, we'll discuss how to apply loops to strings.
Although the examples are short, the techniques work the same whether you have one word or one million words.

\index{charAt}
\index{char}
\index{type!char}

Remember that strings provide a method named \java{charAt}.
It returns a \java{char}, a data type that stores an individual character (as opposed to strings of them):

\begin{code}
String fruit = "banana";
char letter = fruit.charAt(0);
\end{code}

The argument \java{0} means that we want the character at {\bf index} 0.
String indexes range from 0 to $n-1$, where $n$ is the length of the string.
So the character assigned to \java{letter} is \java{'b'}:

\begin{center}
\ttfamily
\begin{tabular}{cccccc}
\hline
\multicolumn{1}{|l|}{b} & \multicolumn{1}{l|}{a} & \multicolumn{1}{l|}{n} & \multicolumn{1}{l|}{a} & \multicolumn{1}{l|}{n} & \multicolumn{1}{l|}{a} \\ \hline
0                       & 1                      & 2                      & 3                      & 4                      & 5
\end{tabular}
\end{center}

We have seen that characters work like the other data types you have seen.
Recall that you can compare them using relational operators:

\begin{code}
if (letter == 'A') {
    System.out.println("It's an A!");
}
\end{code}

\index{quote mark}
\index{escape sequence}

Character literals, like \java{'A'}, appear in single quotes.
Unlike string literals, which appear in double quotes, character literals can contain only a single character.
Escape sequences, like \java{'\\t'}, are legal because they represent a single character.

The increment and decrement operators also work with characters.
So this loop displays the letters of the alphabet:

\begin{code}
System.out.print("Roman alphabet: ");
for (char c = 'A'; c <= 'Z'; c++) {
    System.out.print(c);
}
System.out.println();
\end{code}

The output is shown here:

\begin{stdout}
ABCDEFGHIJKLMNOPQRSTUVWXYZ
\end{stdout}

\index{Unicode}

Java uses {\bf Unicode} to represent characters, so strings can store text in other alphabets like Cyrillic and Greek, and non-alphabetic languages like Chinese.
You can read more about it at the Unicode website (\url{https://unicode.org/}).

In Unicode, each character is represented by a ``code point'', which you can think of as an integer.
The code points for uppercase Greek letters run from 913 to 937, so we can display the Greek alphabet like this:

\begin{code}
System.out.print("Greek alphabet: ");
for (int i = 913; i <= 937; i++) {
    System.out.print((char) i);
}
System.out.println();
\end{code}

This example uses a type cast to convert each integer (in the range) to the corresponding character.
Try running the code and see what happens.


\section{Which Loop to Use}

\java{for} and \java{while} loops have the same capabilities; any \java{for} loop can be rewritten as a \java{while} loop, and vice versa.
For example, we could have printed letters of the alphabet by using a \java{while} loop:

\begin{code}
System.out.print("Roman alphabet: ");
char c = 'A';
while (c <= 'Z') {
    System.out.print(c);
    c++;
}
System.out.println();
\end{code}

You might wonder when to use one or the other.  It depends on whether you know how many times the loop will repeat.

A \java{for} loop is ``definite'', which means we know, at the beginning of the loop, how many times it will repeat.
In the alphabet example, we know it will run 26 times.
In that case, it's better to use a \java{for} loop, which puts all of the loop control code on one line.

A \java{while} loop is ``indefinite'', which means we don't know how many times it will repeat.
For example, when validating user input as in Section~\ref{validate}, it's impossible to know how many times the user will enter a wrong value.
In this case, a \java{while} loop is more appropriate:

\begin{code}
System.out.print("Enter a number: ");
while (!in.hasNextDouble()) {
    String word = in.next();
    System.err.println(word + " is not a number");
    System.out.print("Enter a number: ");
}
double number = in.nextDouble();
\end{code}

It's easier to read the \java{Scanner} method calls when they're not all on one line of code.


\section{String Iteration}

\index{iteration}

Strings provide a method called \java{length} that returns the number of characters in the string.
The following loop iterates the characters in \java{fruit} and displays them, one on each line:

\begin{code}
for (int i = 0; i < fruit.length(); i++) {
    char letter = fruit.charAt(i);
    System.out.println(letter);
}
\end{code}

\index{string!length}
\index{length!string}

Because \java{length} is a method, you have to invoke it with parentheses (there are no arguments).
When \java{i} is equal to the length of the string, the condition becomes \java{false} and the loop terminates.

To find the last letter of a string, you might be tempted to do something like the following:

\begin{code}
int length = fruit.length();
char last = fruit.charAt(length);      // wrong!
\end{code}

\index{StringIndexOutOfBoundsException}
\index{exception!StringIndexOutOfBounds}

This code compiles and runs, but invoking the \java{charAt} method throws a \java{StringIndexOutOfBoundsException}.
The problem is that there is no sixth letter in \java{"banana"}.
Since we started counting at 0, the six letters are indexed from 0 to 5.
To get the last character, you have to subtract 1 from \java{length}:

\begin{code}
int length = fruit.length();
char last = fruit.charAt(length - 1);  // correct
\end{code}

Many string algorithms involve reading one string and building another.
For example, to reverse a string, we can concatenate one character at a time:

\begin{code}
public static String reverse(String s) {
    String r = "";
    for (int i = s.length() - 1; i >= 0; i--) {
        r += s.charAt(i);
    }
    return r;
}
\end{code}

\index{empty string}

The initial value of \java{r} is \java{""}, which is an {\bf empty string}.
The loop iterates the indexes of \java{s} in reverse order.
Each time through the loop, the \java{+=} operator appends the next character to \java{r}.
When the loop exits, \java{r} contains the characters from \java{s} in reverse order.
So the result of \java{reverse("banana")} is \java{"ananab"}.

\section{Array Iteration}

\index{iteration}

You can also iterate through an array using a loop.
For example:

\begin{code}
counts[0] = 7;
counts[1] = counts[0] * 2;
counts[2]++;
counts[3] -= 60;

int i = 0;
while (i < 4) {
    System.out.println(counts[i]);
    i++;
}
\end{code}

This \java{while} loop counts up from 0 to 4.
When \java{i} is 4, the condition fails and the loop terminates.
So the body of the loop is executed only when \java{i} is 0, 1, 2, or 3.
In this context, the variable name \java{i} is short for ``index''.

\index{loop variable}
\index{variable!loop}

Each time through the loop, we use \java{i} as an index into the array, displaying the $i$th element.
This type of array processing is usually written as a \java{for} loop:

\begin{code}
for (int i = 0; i < 4; i++) {
    System.out.println(counts[i]);
}
\end{code}

If the stop condition was \java{<= 4}, rather than \java{< 4}, what do you think
would happen when we try to access the element in the index \java{4} in the loop?

The result would be an \java{ArrayIndexOutOfBoundsException}

You can also use a loop to copy an array:
\begin{code}
double[] b = new double[3];
for (int i = 0; i < 3; i++) {
    b[i] = a[i];
}
\end{code}

The examples so far work only if the array has three elements.
It is better to generalize the code to work with arrays of any size.
We can do that by replacing the magic number, \java{3}, with \java{a.length}:

\begin{code}
double[] b = new double[a.length];
for (int i = 0; i < a.length; i++) {
    b[i] = a[i];
}
\end{code}

\index{search}

Another common pattern is a {\bf search}, which involves traversing an array and ``searching'' for a particular element.
For example, the following method takes an array and a value, and it returns the index where the value appears:

% TODO: comparing floating point values doesn't always work (credit: Fazl)

\begin{code}
public static int search(double[] array, double target) {
    for (int i = 0; i < array.length; i++) {
        if (array[i] == target) {
            return i;
        }
    }
    return -1;  // not found
}
\end{code}

If we find the target value in the array, we return its index immediately.
If the loop exits without finding the target, it returns \java{-1}, a special value chosen to indicate a failed search.
(This code is essentially what the \java{String.indexOf} method does.)

The following code searches an array for the value \java{1.23}, which is the third element.
Because array indexes start at 0, the output is \java{2}:

\begin{code}
double[] array = {3.14, -55.0, 1.23, -0.8};
int index = search(array, 1.23);
System.out.println(index);
\end{code}

\index{reduce}

Another common traversal is a {\bf reduce} operation, which ``reduces'' an array of values down to a single value.
Examples include the sum or product of the elements, the minimum, and the maximum.
The following method takes an array and returns the sum of its elements:

\begin{code}
public static double sum(double[] array) {
    double total = 0.0;
    for (int i = 0; i < array.length; i++) {
        total += array[i];
    }
    return total;
}
\end{code}

\index{accumulator}

Before the loop, we initialize \java{total} to \java{0}.
Each time through the loop, we update \java{total} by adding one element from the array.
At the end of the loop, \java{total} contains the sum of the elements.
A variable used this way is sometimes called an {\bf accumulator}, because it ``accumulates'' the running total.

\section{Building a Histogram}
\label{singlepass}

\index{histogram}
\index{counter}

If these values were exam scores---and they would be pretty bad exam scores in that case---the teacher might present them to the class in the form of a {\bf histogram}.
In statistics, a histogram is a set of counters that keeps track of the number of times each value appears.

For exam scores, we might have 10 counters to keep track of how many students scored in the 90s, the 80s, etc.
To do that, we can traverse the array and count the number of elements that fall in a given range.

The following method takes an array and two integers.
It returns the number of elements that fall in the range from \java{low} to \java{high - 1}:

\begin{code}
public static int inRange(int[] a, int low, int high) {
    int count = 0;
    for (int i = 0; i < a.length; i++) {
        if (a[i] >= low && a[i] < high) {
            count++;
        }
    }
    return count;
}
\end{code}

\index{reduce}

This pattern should look familiar: it is another reduce operation.
Notice that \java{low} is included in the range (\java{>=}), but \java{high} is excluded (\java{<}).
This design keeps us from counting any scores twice.

Now we can count the number of scores in each grade range.
We add the following code to our \java{main} method:

\begin{code}
int[] scores = randomArray(30);
int a = inRange(scores, 90, 100);
int b = inRange(scores, 80, 90);
int c = inRange(scores, 70, 80);
int d = inRange(scores, 60, 70);
int f = inRange(scores, 0, 60);
\end{code}

This code is repetitive, but it is acceptable as long as the number of ranges is small.
Suppose we wanted to keep track of the number of times each individual score appears.
Then we would have to write 100 lines of code:

\begin{code}
int count0 = inRange(scores, 0, 1);
int count1 = inRange(scores, 1, 2);
int count2 = inRange(scores, 2, 3);
...
int count99 = inRange(scores, 99, 100);
\end{code}

What we need is a way to store 100 counters, preferably so we can use an index to access them.
Wait a minute---that's exactly what an array does.

The following fragment creates an array of 100 counters, one for each possible score.
It loops through the scores and uses \java{inRange} to count how many times each score appears.
Then it stores the results in the \java{counts} array:

\begin{code}
int[] counts = new int[100];
for (int i = 0; i < counts.length; i++) {
    counts[i] = inRange(scores, i, i + 1);
}
\end{code}

Notice that we are using the loop variable \java{i} three times: as an index into the \java{counts} array, and in the last two arguments of \java{inRange}.

\index{efficiency}

The code works, but it is not as efficient as it could be.
Every time the loop invokes \java{inRange}, it traverses the entire array.
It would be better to make a single pass through the \java{scores} array.

For each score, we already know which range it falls in---the score itself.
We can use that value to increment the corresponding counter.
This code traverses the array of scores {\em only once} to generate the histogram:

\begin{code}
int[] counts = new int[100];
for (int i = 0; i < scores.length; i++) {
    int index = scores[i];
    counts[index]++;
}
\end{code}

Each time through the loop, it selects one element from \java{scores} and uses it as an index to increment the corresponding element of \java{counts}.
Because this code traverses the array of scores only once, it is much more efficient.

\section{The Enhanced for Loop}
\label{enhanced}

Since traversing arrays is so common, Java provides an alternative syntax that makes the code more compact.
Consider a \java{for} loop that displays the elements of an array on separate lines:

\begin{code}
for (int i = 0; i < values.length; i++) {
    int value = values[i];
    System.out.println(value);
}
\end{code}

We could rewrite the loop like this:

\begin{code}
for (int value : values) {
    System.out.println(value);
}
\end{code}

\index{enhanced for loop}
\index{for!enhanced}

This statement is called an {\bf enhanced for loop}, also known as the ``for each'' loop.
You can read the code as, ``for each \java{value} in \java{values}''.
It's conventional to use plural nouns for array variables and singular nouns for element variables.

Notice how the single line \java{for (int value : values)} replaces the first two lines of the standard \java{for} loop.
It hides the details of iterating each index of the array, and instead, focuses on the values themselves.

Using the enhanced \java{for} loop, and removing the temporary variable, we can write the histogram code from the previous section more concisely:

\begin{code}
int[] counts = new int[100];
for (int score : scores) {
    counts[score]++;
}
\end{code}

Enhanced \java{for} loops often make the code more readable, especially for accumulating values.
But they are not helpful when you need to refer to the index, as in search operations:

\begin{code}
for (double d : array) {
    if (d == target) {
        // array contains d, but we don't know where
    }
}
\end{code}



\section{Vocabulary}

\begin{description}

\term{loop}
A statement that executes a sequence of statements repeatedly.

\term{loop body}
The statements inside the loop.

\term{infinite loop}
A loop whose condition is always true.

\term{iterative}
A method or algorithm that repeats steps by using one or more loops.

\term{increment}
Increase the value of a variable.

\term{decrement}
Decrease the value of a variable.

\term{iteration}
Executing a sequence of statements repeatedly.

\term{loop variable}
A variable that is initialized, tested, and updated in order to control a loop.

\end{description}


\section{Exercises}

The code for this chapter is in the {\it ch06} directory of {\it ThinkJavaCode2}.
See page~\pageref{code} for instructions on how to download the repository.
Before you start the exercises, we recommend that you compile and run the examples.

If you have not already read Appendix~\ref{debugger}, now might be a good time.
It describes the DrJava debugger, which is a useful tool for visualizing the flow of execution through loops.


\begin{exercise}  %%V6 Ex7.1

Consider the following methods (\java{main} and \java{loop}):

\begin{enumerate}

\item Draw a table that shows the value of the variables \java{i} and \java{n} during the execution of \java{loop}.
The table should contain one column for each variable and one line for each iteration.

\item What is the output of this program?

\item Can you prove that this loop terminates for any positive value of \java{n}?

% If i is odd and we increment by 1, the result is even.  So the second
% branch is always followed by the first branch.
% If i is even and we divide by 2, the result might be odd.  So in the
% worst case, we might alternate between the branches.
% But we can't do more of the second branch than the first.
% So we divide at least as often as we add.

% If i is 1, we're done.
% If i is 2, we divide by 2 and we're done.
% If i is greater than 2, the first branch decreases more than the
% second branch increases.
% So if we do one of each, the net effect is a decrease.
% Therefore, the value of i has to decrease after any two steps.

\end{enumerate}

\begin{code}
public static void main(String[] args) {
    loop(10);
}

public static void loop(int n) {
    int i = n;
    while (i > 1) {
        System.out.println(i);
        if (i % 2 == 0) {
            i = i / 2;
        } else {
            i = i + 1;
        }
    }
}
\end{code}

\end{exercise}


\begin{exercise}  %%V6 Ex7.2

Let's say you are given a number, $a$, and you want to find its square root.
One way to do that is to start with a rough guess about the answer, $x_0$, and then improve the guess by using this formula:
%
\[ x_1 =(x_0 + a/x_0) / 2 \]
%
For example, if we want to find the square root of 9, and we start with $x_0 = 6$, then $x_1 = (6 + 9/6) / 2 = 3.75$, which is closer.
We can repeat the procedure, using $x_1$ to calculate $x_2$, and so on.
In this case, $x_2 = 3.075$ and $x_3 = 3.00091$.
So the repetition converges quickly on the correct answer.

Write a method called \java{squareRoot} that takes a \java{double} and returns an approximation of the square root of the parameter, using this technique.
You should not use \java{Math.sqrt}.

As your initial guess, you should use $a/2$.
Your method should iterate until it gets two consecutive estimates that differ by less than 0.0001.
%In other words, return when the absolute value of $x_n - x_{n-1}$ is less than 0.0001.
You can use \java{Math.abs} to calculate the absolute value of the difference.

\end{exercise}


\begin{exercise}  %%V6 Ex7.6

One way to evaluate $\exp(-x^2)$ is to use the infinite series expansion:
%
\[ \exp(-x^2) = 1 - x^2 + x^4/2 - x^6/6 + \ldots \]
%
The $i$th term in this series is $(-1)^i x^{2i} / i!$.
Write a method named \java{gauss} that takes \java{x} and \java{n} as arguments and returns the sum of the first \java{n} terms of the series.
You should not use \java{factorial} or \java{pow}.

\end{exercise}


\begin{exercise}  %%V6 Ex9.5

\index{abecedarian}

A word is said to be ``abecedarian'' if the letters in the word appear in alphabetical order.
For example, the following are all six-letter English abecedarian words:

\begin{quote}
abdest, acknow, acorsy, adempt, adipsy, agnosy, befist, behint, %\\
beknow, bijoux, biopsy, cestuy, chintz, deflux, dehors, dehort, %\\
deinos, diluvy, dimpsy %\\
\end{quote}

Write a method called \java{isAbecedarian} that takes a \java{String} and returns a \java{boolean} indicating whether the word is abecedarian.
%Your method can be iterative or recursive.

\end{exercise}


\begin{exercise}  %%V6 Ex9.6
\label{doubloon}

\index{doubloon}

A word is said to be a ``doubloon'' if every letter that appears in the word appears exactly twice.
Here are some example doubloons found in the dictionary:

\begin{quote}
Abba, Anna, appall, appearer, appeases, arraigning, beriberi, bilabial, boob, Caucasus, coco, Dada, deed, Emmett, Hannah, horseshoer, intestines, Isis, mama, Mimi, murmur, noon, Otto, papa, peep, reappear, redder, sees, Shanghaiings, Toto
\end{quote}

Write a method called \java{isDoubloon} that takes a string and checks whether it is a doubloon.
To ignore case, invoke the \java{toLowerCase} method before checking.
\end{exercise}


\begin{exercise}  %%V6 Ex9.8

\index{Scrabble}

In Scrabble\footnote{Scrabble is a registered trademark owned in the USA and Canada by Hasbro Inc., and in the rest of the world by J.\ W.\ Spear \& Sons Limited of Maidenhead, Berkshire, England, a subsidiary of Mattel Inc.} each player has a set of tiles with letters on them.
The object of the game is to use those letters to spell words.
The scoring system is complex, but longer words are usually worth more than shorter words.

% Full credit to Marc Loy for getting this reference.

Imagine you are given your set of tiles as a string, like \java{"quijibo"}, and you are given another string to test, like \java{"jib"}.

Write a method called \java{canSpell} that takes two strings and checks whether the set of tiles can spell the word.
You might have more than one tile with the same letter, but you can use each tile only once.

\end{exercise}
