\chapter{Objects: Strings} 
\label{object:Strings}

\index{object-oriented}

Java is an {\bf object-oriented} language, which means that it uses objects to (1)~represent data and (2)~provide methods related to them.
This way of organizing programs is a powerful design concept, and we will introduce it gradually throughout the remainder of the book.

\index{object}
\index{System.in}
\index{System.out}

An {\bf object} is a collection of data that provides a set of methods.
For example, \java{Scanner}, which you saw in Section~\ref{scanner}, is an object that provides methods for parsing input.
\java{System.out} and \java{System.in} are also objects.

Strings are objects, too.
They contain characters and provide methods for manipulating character data.
Other data types, like \java{Integer}, contain numbers and provide methods for manipulating number data.
We will explore some of these methods in this chapter.


\section{Primitives vs Objects}

\index{primitive}

Not everything in Java is an object: \java{int}, \java{double}, \java{char}, and \java{boolean} are {\bf primitive} types.
%We will explain some of the differences between object types and primitive types as we go along.
When you declare a variable with a primitive type, Java reserves a small amount of memory to store its value.
Figure~\ref{fig.mem1} shows how the following values are stored in memory:

\begin{code}
int number = -2;
char symbol = '!';
\end{code}

\begin{figure}[!ht]
\begin{center}
\includegraphics{figs/mem1.pdf}
\caption{Memory diagram of two primitive variables.}
\label{fig.mem1}
\end{center}
\end{figure}

\index{memory diagram}
\index{diagram!memory}


Objects work in a similar way.
For example, this line declares a \java{String} variable named \java{word} and creates a \java{String} object, as shown in Figure~\ref{fig.mem3}:

\begin{code}
String word = "dog";
\end{code}

\begin{figure}[!ht]
\begin{center}
\includegraphics{figs/mem3.pdf}
\caption{Memory diagram of a \java{String} object.}
\label{fig.mem3}
\end{center}
\end{figure}

Objects and arrays are usually created with the \java{new} keyword, which allocates memory for them. %(see Section~\ref{allocate})
For convenience, you don't have to use \java{new} to create strings:

\begin{code}
String word1 = new String("dog");  // creates a string object
String word2 = "dog";   // implicitly creates a string object
\end{code}

\index{string!comparing}

Recall from Section~\ref{strcmp} that you need to use the \java{equals} method to compare strings.
The \java{equals} method traverses the \java{String} objects and tests whether they contain the same characters.

To test whether two integers or other primitive types are equal, you can simply use the \java{==} operator.
But two \java{String} objects with the same characters would not be considered equal in the \java{==} sense.
The \java{==} operator, when applied to string variables, tests only whether they refer to the {\em same} object.

As mentioned above, Strings contain characters, and since they are objects, they have methods for manipulating
character data. The rest of the chapter will cover some String methods.

\section{Characters}
Some of the most interesting problems in computer science involve searching and manipulating text.
\index{charAt}
\index{char}
\index{type!char}
Strings provide a method named \java{charAt}.
It returns a \java{char}, a data type that stores an individual character (as opposed to strings of them):


\begin{code}
String fruit = "banana";
char letter = fruit.charAt(0);
\end{code}

The argument \java{0} means that we want the character at {\bf index} 0.
String indexes range from 0 to $n-1$, where $n$ is the length of the string.
So the character assigned to \java{letter} is \java{'b'}:

\begin{center}
\ttfamily
\begin{tabular}{cccccc}
\hline
\multicolumn{1}{|l|}{b} & \multicolumn{1}{l|}{a} & \multicolumn{1}{l|}{n} & \multicolumn{1}{l|}{a} & \multicolumn{1}{l|}{n} & \multicolumn{1}{l|}{a} \\ \hline
0                       & 1                      & 2                      & 3                      & 4                      & 5
\end{tabular}
\end{center}

Characters work like the other data types you have seen.
You can compare them using relational operators:

\begin{code}
if (letter == 'A') {
    System.out.println("It's an A!");
}
\end{code}

\index{quote mark}
\index{escape sequence}

Character literals, like \java{'A'}, appear in single quotes.
Unlike string literals, which appear in double quotes, character literals can contain only a single character.
Escape sequences, like \java{'\\t'}, are legal because they represent a single character.

Underneath the hood, characters have numerical values.
So if we called \java{incrementLetter('A')},
based on the following recursive method, 
the letters of the alphabet will be displayed:

\begin{code}
public static void incrementLetter(char c) {
    if (c > 'Z') {
      return;
    }
    System.out.print(c);
    incrementLetter(c+1);
}
\end{code}

The output is shown here:

\begin{stdout}
ABCDEFGHIJKLMNOPQRSTUVWXYZ
\end{stdout}

\index{Unicode}

Java uses {\bf Unicode} to represent characters, so strings can store text in other alphabets like Cyrillic and Greek, and non-alphabetic languages like Chinese.
You can read more about it at the Unicode website (\url{https://unicode.org/}).

In Unicode, each character is represented by a ``code point'', which you can think of as an integer.
The code points for uppercase Greek letters run from 913 to 937, so we can display the Greek alphabet like this:

\begin{code}
public static void printGreek(int i) {
    if (i >= 938) {
      return;
    }
    System.out.print((char) i);
    printGreek(i+1);
}

System.out.print("Greek alphabet: ");
printGreek(913);
\end{code}

This example uses a type cast to convert each integer (in the range) to the corresponding character.
Try running the code and see what happens.


\section{String Iteration}

\index{iteration}

Strings provide a method called \java{length} that returns the number of characters in the string.
The following method displays the characters in \java{fruit}, one on each line:

\begin{code}
public static void displayLetters(String fruit, int index) {
    if (index == fruit.length()) {
        return;
    }
    char letter = fruit.charAt(index);
    System.out.println(letter);
    displayLetters(fruit, index+1);
}

public static void displayLetters(String fruit) {
    displayLetters(fruit, 0);
}

displayLetters("orange");
\end{code}

\index{string!length}
\index{length!string}

Because \java{length} is a method, you have to invoke it with parentheses (there are no arguments).
When \java{i} is equal to the length of the string, the condition becomes \java{false} and 
the recursive method displayLetters stops calling itself.

To find the last letter of a string, you might be tempted to do something like the following:

\begin{code}
int length = fruit.length();
char last = fruit.charAt(length);      // wrong!
\end{code}

\index{StringIndexOutOfBoundsException}
\index{exception!StringIndexOutOfBounds}

This code compiles and runs, but invoking the \java{charAt} method throws a \java{StringIndexOutOfBoundsException}.
The problem is that there is no sixth letter in \java{"banana"}.
Since we started counting at 0, the six letters are indexed from 0 to 5.
To get the last character, you have to subtract 1 from \java{length}:

\begin{code}
int length = fruit.length();
char last = fruit.charAt(length - 1);  // correct
\end{code}

Many string algorithms involve reading one string and building another.
For example, to reverse a string, we can concatenate one character at a time:

\begin{code}
public static String reverse(String s, int index) {
    if (index < 0) {
      return "";
    }
    return s.charAt(index) + reverse(s, index-1);
}

public static String reverse(String s) {
    return reverse(s, s.length() - 1);
}
\end{code}

\index{empty string}

The initial value of \java{r} is \java{""}, which is an {\bf empty string}.
The loop iterates the indexes of \java{s} in reverse order.
Each time through the loop, the \java{+=} operator appends the next character to \java{r}.
When the loop exits, \java{r} contains the characters from \java{s} in reverse order.
So the result of \java{reverse("banana")} is \java{"ananab"}.


\section{The indexOf Method}

\index{indexOf}

To search for a specific character in a string, you could write a recursive method and use \java{charAt} as in the previous section.
However, the \java{String} class already provides a method for doing just that:

\begin{code}
String fruit = "banana";
int index = fruit.indexOf('a');     // returns 1
\end{code}

This example finds the index of \java{'a'} in the string.
But the letter appears three times, so it's not obvious what \java{indexOf} might do.
According to the documentation, it returns the index of the {\em first} appearance.

To find subsequent appearances, you can use another version of \java{indexOf}, which takes a second argument that indicates where in the string to start looking:

\begin{code}
int index = fruit.indexOf('a', 2);  // returns 3
\end{code}

To visualize how \java{indexOf} and other \java{String} methods work, it helps to draw a picture like Figure~\ref{fig.banana}.
The previous code starts at index 2 (the first \java{'n'}) and finds the next \java{'a'}, which is at index 3.

\index{memory diagram}
\index{diagram!memory}

\begin{figure}[!ht]
\begin{center}
\includegraphics{figs/banana.pdf}
\caption{Memory diagram for a \java{String} of six characters.}
\label{fig.banana}
\end{center}
\end{figure}

%\begin{center}
%\begin{tabular}{c|c|c|c|c|c}
%%\hline
%b & a & n & a & n & a \\
%\hline
%0 & 1 & 2 & 3 & 4 & 5 \\
%%\hline
%\end{tabular}
%\end{center}

If the character happens to appear at the starting index, the starting index is the answer.
So \java{fruit.indexOf('a', 5)} returns \java{5}.
If the character does not appear in the string, \java{indexOf} returns \java{-1}.
Since indexes cannot be negative, this value indicates the character was not found.

You can also use \java{indexOf} to search for an entire string, not just a single character.
For example, the expression \java{fruit.indexOf("nan")} returns \java{2}.


\section{Substrings}

\index{substring}

In addition to searching strings, we often need to extract parts of strings.
The \java{substring} method returns a new string that copies letters from an existing string, given a pair of indexes:

\begin{itemize}
\item \java{fruit.substring(0, 3)} returns \java{"ban"}
\item \java{fruit.substring(2, 5)} returns \java{"nan"}
\item \java{fruit.substring(6, 6)} returns \java{""}
\end{itemize}

Notice that the character indicated by the second index is {\em not} included.
Defining \java{substring} this way simplifies some common operations.
For example, to select a substring with length \java{len}, starting at index \java{i}, you could write \java{fruit.substring(i, i + len)}.
%So \java{fruit.substring(2, 2 + 3)} returns \java{"nan"}.

\index{overloaded}

Like most string methods, \java{substring} is {\bf overloaded}.
That is, there are other versions of \java{substring} that have different parameters.
If it's invoked with one argument, it returns the letters from that index to the end:

\begin{itemize}
\item \java{fruit.substring(0)} returns \java{"banana"}
\item \java{fruit.substring(2)} returns \java{"nana"}
\item \java{fruit.substring(6)} returns \java{""}
\end{itemize}

The first example returns a copy of the entire string.
The second example returns all but the first two characters.
As the last example shows, \java{substring} returns the empty string if the argument is the length of the string.

We could also use \java{fruit.substring(2, fruit.length())} to get the result \java{"nana"}.
But calling \java{substring} with one argument is more convenient when you want the end of the string.


\section{String Comparison}
\label{strcmp}

\index{equals}
\index{string!comparing}

When comparing strings, it might be tempting to use the \java{==} and \java{!=} operators.
But that will almost never work.
The following code compiles and runs, but it always displays {\tt Goodbye!}\ regardless what the user types.

\begin{code}
System.out.print("Play again? ");
String answer = in.nextLine();
if (answer == "yes") {                 // wrong!
    System.out.println("Let's go!");
} else {
    System.out.println("Goodbye!");
}
\end{code}

The problem is that the \java{==} operator checks whether the two operands refer to the {\em same object}.
Even if the answer is \java{"yes"}, it will refer to a different object in memory than the literal string \java{"yes"} in the code.
You'll learn more about objects and references in the next chapter.

The correct way to compare strings is with the \java{equals} method, like this:

\begin{code}
if (answer.equals("yes")) {
    System.out.println("Let's go!");
}
\end{code}

This example invokes \java{equals} on \java{answer} and passes \java{"yes"} as an argument.
The \java{equals} method returns \java{true} if the strings contain the same characters; otherwise, it returns \java{false}.

\index{compareTo}

If two strings differ, we can use \java{compareTo} to see which comes first in alphabetical order:

\begin{code}
String name1 = "Alan Turing";
String name2 = "Ada Lovelace";
int diff = name1.compareTo(name2);
if (diff < 0) {
    System.out.println("name1 comes before name2.");
} else if (diff > 0) {
    System.out.println("name2 comes before name1.");
} else {
    System.out.println("The names are the same.");
}
\end{code}

The return value from \java{compareTo} is the difference between the first characters in the strings that are not the same.
In the preceding code, \java{compareTo} returns positive \java{8}, because the second letter of \java{"Ada"} comes before the second letter of \java{"Alan"} by eight letters.

If the first string (the one on which the method is invoked) comes earlier in the alphabet, the difference is negative.
If it comes later in the alphabet, the difference is positive.
If the strings are equal, their difference is zero.

\index{case-sensitive}

Both \java{equals} and \java{compareTo} are case-sensitive.
In Unicode, uppercase letters come before lowercase letters.
So \java{"Ada"} comes before \java{"ada"}.


\section{String Formatting}

\index{printf}

In Section~\ref{printf}, we learned how to use \java{System.out.printf} to display formatted output.
Sometimes programs need to create strings that are formatted a certain way, but not display them immediately (or ever).
For example, the following method returns a time string in 12-hour format:

\begin{code}
public static String timeString(int hour, int minute) {
    String ampm;
    if (hour < 12) {
        ampm = "AM";
        if (hour == 0) {
            hour = 12;  // midnight
        }
    } else {
        ampm = "PM";
        hour = hour - 12;
    }
    return String.format("%02d:%02d %s", hour, minute, ampm);
}
\end{code}

\index{string!format}

\java{String.format} takes the same arguments as \java{System.out.printf}: a format specifier followed by a sequence of values.
The main difference is that \java{System.out.printf} {\em displays} the result on the screen.
\java{String.format} creates a new string but does not display anything.

In this example, the format specifier \java{\%02d} means ``two-digit integer padded with zeros'', so \java{timeString(19, 5)} returns the string \java{"07:05 PM"}.
As an exercise, try writing two nested \java{for} loops (in \java{main}) that invoke \java{timeString} and display all possible times over a 24-hour period.

Be sure to skim through the documentation for \java{String}.
Knowing what other methods are there will help you avoid reinventing the wheel.
The easiest way to find documentation for Java classes is to do a web search for ``Java'' and the name of the class.


\section{Vocabulary}

\begin{description}

\term{increment}
Increase the value of a variable.

\term{decrement}
Decrease the value of a variable.

\term{index}
An integer variable or value used to indicate a character in a string.

\term{Unicode}
An international standard for representing characters in most of the world's languages.

\term{empty string}
The string \java{""}, which contains no characters and has a length of zero.

\term{overloaded}
Two or more methods with the same name but different parameters.

\end{description}


\section{Exercises}

The code for this chapter is in the {\it ch06} directory of {\it ThinkJavaCode2}.
See page~\pageref{code} for instructions on how to download the repository.
Before you start the exercises, we recommend that you compile and run the examples.

If you have not already read Appendix~\ref{debugger}, now might be a good time.
It describes the DrJava debugger, which is a useful tool for visualizing the flow of execution through loops.


\begin{exercise}  %%V6 Ex7.1

Consider the following methods (\java{main} and \java{loop}):

\begin{enumerate}

\item Draw a table that shows the value of the variables \java{i} and \java{n} during the execution of \java{loop}.
The table should contain one column for each variable and one line for each iteration.

\item What is the output of this program?

\item Can you prove that this loop terminates for any positive value of \java{n}?

% If i is odd and we increment by 1, the result is even.  So the second
% branch is always followed by the first branch.
% If i is even and we divide by 2, the result might be odd.  So in the
% worst case, we might alternate between the branches.
% But we can't do more of the second branch than the first.
% So we divide at least as often as we add.

% If i is 1, we're done.
% If i is 2, we divide by 2 and we're done.
% If i is greater than 2, the first branch decreases more than the
% second branch increases.
% So if we do one of each, the net effect is a decrease.
% Therefore, the value of i has to decrease after any two steps.

\end{enumerate}

\begin{code}
public static void main(String[] args) {
    loop(10);
}

public static void loop(int n) {
    int i = n;
    while (i > 1) {
        System.out.println(i);
        if (i % 2 == 0) {
            i = i / 2;
        } else {
            i = i + 1;
        }
    }
}
\end{code}

\end{exercise}


\begin{exercise}  %%V6 Ex7.2

Let's say you are given a number, $a$, and you want to find its square root.
One way to do that is to start with a rough guess about the answer, $x_0$, and then improve the guess by using this formula:
%
\[ x_1 =(x_0 + a/x_0) / 2 \]
%
For example, if we want to find the square root of 9, and we start with $x_0 = 6$, then $x_1 = (6 + 9/6) / 2 = 3.75$, which is closer.
We can repeat the procedure, using $x_1$ to calculate $x_2$, and so on.
In this case, $x_2 = 3.075$ and $x_3 = 3.00091$.
So the repetition converges quickly on the correct answer.

Write a method called \java{squareRoot} that takes a \java{double} and returns an approximation of the square root of the parameter, using this technique.
You should not use \java{Math.sqrt}.

As your initial guess, you should use $a/2$.
Your method should iterate until it gets two consecutive estimates that differ by less than 0.0001.
%In other words, return when the absolute value of $x_n - x_{n-1}$ is less than 0.0001.
You can use \java{Math.abs} to calculate the absolute value of the difference.

\end{exercise}


\begin{exercise}  %%V6 Ex7.6

One way to evaluate $\exp(-x^2)$ is to use the infinite series expansion:
%
\[ \exp(-x^2) = 1 - x^2 + x^4/2 - x^6/6 + \ldots \]
%
The $i$th term in this series is $(-1)^i x^{2i} / i!$.
Write a method named \java{gauss} that takes \java{x} and \java{n} as arguments and returns the sum of the first \java{n} terms of the series.
You should not use \java{factorial} or \java{pow}.

\end{exercise}


\begin{exercise}  %%V6 Ex9.5

\index{abecedarian}

A word is said to be ``abecedarian'' if the letters in the word appear in alphabetical order.
For example, the following are all six-letter English abecedarian words:

\begin{quote}
abdest, acknow, acorsy, adempt, adipsy, agnosy, befist, behint, %\\
beknow, bijoux, biopsy, cestuy, chintz, deflux, dehors, dehort, %\\
deinos, diluvy, dimpsy %\\
\end{quote}

Write a method called \java{isAbecedarian} that takes a \java{String} and returns a \java{boolean} indicating whether the word is abecedarian.
%Your method can be iterative or recursive.

\end{exercise}


\begin{exercise}  %%V6 Ex9.6
\label{doubloon}

\index{doubloon}

A word is said to be a ``doubloon'' if every letter that appears in the word appears exactly twice.
Here are some example doubloons found in the dictionary:

\begin{quote}
Abba, Anna, appall, appearer, appeases, arraigning, beriberi, bilabial, boob, Caucasus, coco, Dada, deed, Emmett, Hannah, horseshoer, intestines, Isis, mama, Mimi, murmur, noon, Otto, papa, peep, reappear, redder, sees, Shanghaiings, Toto
\end{quote}

Write a method called \java{isDoubloon} that takes a string and checks whether it is a doubloon.
To ignore case, invoke the \java{toLowerCase} method before checking.
\end{exercise}


\begin{exercise}  %%V6 Ex9.8

\index{Scrabble}

In Scrabble\footnote{Scrabble is a registered trademark owned in the USA and Canada by Hasbro Inc., and in the rest of the world by J.\ W.\ Spear \& Sons Limited of Maidenhead, Berkshire, England, a subsidiary of Mattel Inc.} each player has a set of tiles with letters on them.
The object of the game is to use those letters to spell words.
The scoring system is complex, but longer words are usually worth more than shorter words.

% Full credit to Marc Loy for getting this reference.

Imagine you are given your set of tiles as a string, like \java{"quijibo"}, and you are given another string to test, like \java{"jib"}.

Write a method called \java{canSpell} that takes two strings and checks whether the set of tiles can spell the word.
You might have more than one tile with the same letter, but you can use each tile only once.

\end{exercise}
