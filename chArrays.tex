\chapter{Arrays and References}

Up to this point, the only variables we have used were for individual values such as numbers or strings.
In this chapter, you'll learn how to store multiple values of the same type by using a single variable.
This language feature will enable you to write programs that manipulate larger amounts of data.

For example, Exercise~\ref{doubloon} asked you to check whether every letter in a string appears exactly twice.
One algorithm (which hopefully you already discovered) looks at every character in the string 26 times, once for each lowercase letter:

\begin{code}
public static boolean isDoubloon(String str, char Char) {
    if (Char > 'z') {
        return true;
    }
    int firstIndex = str.indexOf(Char);
    if (firstIndex == -1 || firstIndex == str.length() - 1) {
       return isDoubloon(str, (char) (Char+1));
    }
    int secondIndex = str.indexOf(Char, firstIndex+1);
    if (secondIndex == -1) {
      return false; // the character appeared once
    }

    int thirdIndex = str.indexOf(Char, secondIndex+1);
    if (thirdIndex == -1) {
      return isDoubloon(str, (char) (Char+1));
    }

    return false; // if  the count is not 0 or 2, return false
}

public static boolean isDoubloon(String str) {
  return isDoubloon(str, 'a');

}
\end{code}

This approach is inefficient, especially when the string is long.  For example, there are more than 3 million characters in {\it War and Peace}; to process the whole book, this approach would run about 80 million times.

Another algorithm would initialize 26 variables to zero, go through the string {\em one time}, and use a giant \java{if} statement to update the variable for each letter.
But who wants to declare 26 variables?

That's where arrays come in.
We can use a single variable to store 26 integers.
Rather than use an \java{if} statement to update each value, we can use arithmetic to update the $n$th value directly.
We will present this algorithm at the end of the chapter.


\section{Creating Arrays}

\index{array}
\index{element}

An {\bf array} is a sequence of values; the values in the array are called {\bf elements}.
You can make an array of \java{int}s, \java{double}s, \java{String}s, or any other type, but all the values in an array must have the same type.

\index{type!array}
\index{[ ] square brackets}
\index{brackets!square}

To create an array, you have to declare a variable with an {\em array type} and then create the array itself.
Array types look like other Java types, except they are followed by square brackets (\java{[]}).
For example, the following lines declare that \java{counts} is an ``integer array'' and \java{values} is a ``double array'':

\begin{code}
int[] counts;
double[] values;
\end{code}

\index{new}
\index{operator!new}
\index{allocate}

To create the array itself, you have to use the \java{new} operator, which you first saw in Section~\ref{scanner}.
The \java{new} operator {\bf allocates} memory for the array and automatically initializes all of its elements to zero:

\begin{code}
counts = new int[4];
values = new double[size];
\end{code}

The first assignment makes \java{counts} refer to an array of four integers.
The second makes \java{values} refer to an array of \java{double}s, but the number of elements depends on the value of \java{size} (at the time the array is created).

Of course, you can also declare the variable and create the array with a single line of code:

\begin{code}
int[] counts = new int[4];
double[] values = new double[size];
\end{code}

\index{NegativeArraySizeException}
\index{exception!NegativeArraySize}

You can use any integer expression for the size of an array, as long as the value is nonnegative.
If you try to create an array with \java{-4} elements, for example, you will get a \java{NegativeArraySizeException}.
An array with zero elements is allowed, and there are special uses for such arrays.

You can initialize an array with a comma-separated sequence of elements enclosed in braces, like this:

\begin{code}
int[] a = {1, 2, 3, 4};
\end{code}

This statement creates an array variable, \java{a}, and makes it refer to an array with four elements.


\section{Accessing Elements}
\label{elements}

When you create an array with the \java{new} operator, the elements are initialized to zero.
Figure~\ref{fig.array} shows a memory diagram of the \java{counts} array so far.

\begin{figure}[!ht]
\begin{center}
\includegraphics{figs/array.pdf}
\caption{Memory diagram of an \java{int} array.}
\label{fig.array}
\end{center}
\end{figure}

\index{reference}

The arrow indicates that the value of \java{counts} is a {\bf reference} to the array.
You should think of {\em the array} and {\em the variable} that refers to it as two different things.
As you'll soon see, we can assign a different variable to refer to the same array, and we can change the value of \java{counts} to refer to a different array.

\index{element}
\index{index}
\index{array!element}
\index{array!index}

The boldface numbers inside the boxes are the elements of the array.
The lighter numbers outside the boxes are the {\bf indexes} used to identify each location in the array.
As with strings, the index of the first element is 0, not 1.
For this reason, we sometimes refer to the first element as the ``zeroth'' element.

The \java{[]} operator selects elements from an array:

\begin{code}
System.out.println("The zeroth element is " + counts[0]);
\end{code}

You can use the \java{[]} operator anywhere in an expression:

\begin{code}
counts[0] = 7;
counts[1] = counts[0] * 2;
counts[2]++;
counts[3] -= 60;
\end{code}

Figure~\ref{fig.array2} shows the result of these statements.

\begin{figure}[!ht]
\begin{center}
\includegraphics{figs/array2.pdf}
\caption{Memory diagram after several assignment statements.}
\label{fig.array2}
\end{center}
\end{figure}

You can use any expression as an index, as long as it has type \java{int}.
\index{ArrayIndexOutOfBoundsException}
\index{exception!ArrayIndexOutOfBounds}

For the \java{counts} array, the only legal indexes are 0, 1, 2, and 3.
If the index is negative or greater than 3, the result is an \java{ArrayIndexOutOfBoundsException}.

\section{Displaying Arrays}
\label{printarray}

\index{array!printing}

You can use \java{println} to display an array, but it probably doesn't do what you would like.
For example, say you print an array like this:

\begin{code}
int[] a = {1, 2, 3, 4};
System.out.println(a);
\end{code}

The output is something like this:

\begin{stdout}
[I@bf3f7e0
\end{stdout}

The bracket indicates that the value is an array, \java{I} stands for ``integer'', and the rest represents the address of the array in memory.

If we want to display the elements of the array, we can do it ourselves
by writing a method that prints out the value at one index and then
recursively calls itself with the next index.

\index{printArray}

\begin{code}
public static void printArray(int[] array, int index) {
  if (index == array.length) {
    return;
  }
  System.out.print(array[index] );
  printArray(array, index + 1);
}

public static void printArray(int[] array) {
  printArray(array, 0);
}
\end{code}

Given the previous array, the output of \java{printArray} is as follows:

\begin{stdout}
{1 2 3 4 }
\end{stdout}

\index{utility class}
\index{Arrays class}

The Java library includes a class, \java{java.util.Arrays}, that provides methods for working with arrays.
One of them, \java{toString}, returns a string representation of an array.
After importing \java{Arrays}, we can invoke \java{toString} like this:

\begin{code}
System.out.println(Arrays.toString(a));
\end{code}

And the output is shown here:

\begin{stdout}
[1, 2, 3, 4]
\end{stdout}

Notice that \java{Arrays.toString} uses square brackets instead of curly braces.
But it beats writing your own \java{printArray} method.


\section{Copying Arrays}

\index{array!copying}

As explained in Section~\ref{elements}, array variables contain {\em references} to arrays.
When you make an assignment to an array variable, it simply copies the reference.
But it doesn't copy the array itself.
For example:

\begin{code}
double[] a = new double[3];
double[] b = a;
\end{code}

These statements create an array of three \java{double}s and make two different variables refer to it, as shown in Figure~\ref{fig.array3}.

\index{memory diagram}
\index{diagram!memory}

\begin{figure}[!ht]
\begin{center}
\includegraphics{figs/array3.pdf}
\caption{Memory diagram of two variables referring to the same array.}
\label{fig.array3}
\end{center}
\end{figure}

\index{alias}

Any changes made through either variable will be seen by the other.
For example, if we set \java{a[0] = 17.0}, and then display \java{b[0]}, the result is {\tt 17.0}.
Because \java{a} and \java{b} are different names for the same thing, they are sometimes called {\bf aliases}.

If you actually want to copy the array, not just the reference, you have to create a new array and copy the elements from one to the other, like this:

\begin{code}

public static void copyArrayIndexValue(double[] orig, double[] duplicate, int index) {
  if (index == orig.length) {
    return
  }
  duplicate[index] = orig[index];
  copyArrayIndexValue(orig, duplicate, index + 1);

}

public static void copyArray(double[] orig, double[] duplicate) {
  copyArrayIndexValue(orig, duplicate, 0);
} 

double[] b = new double[3];
copyArray(a, b);
\end{code}

%\section{Array Length}
\index{length!array}
\index{array!length}

Using \java{orig.length} in the example above allows us 
to generalize the code to work with arrays of any size.

All arrays have a built-in constant, \java{length}, that stores the number of elements.
In contrast to \java{String.length()}, which is a method, \java{a.length} is a constant.
The expression \java{a.length} may look like a method invocation, but there are no parentheses and no arguments.

In the last frame when the method \java{copyArrayIndexValue()}
is invoked, \java{index} is \java{orig.length - 1}, which is one more than the index of the last element.
In this condition, we do not try to access \java{a[orig.length]}, that would throw an exception.

\index{Arrays class}

\java{java.util.Arrays} provides a method named \java{copyOf} that performs this task for you.
So you can replace the previous code with one line:

\begin{code}
double[] b = Arrays.copyOf(a, 3);
\end{code}

The second parameter is the number of elements you want to copy, so \java{copyOf} can also be used to copy part of an array.
Figure~\ref{fig.array4} shows the state of the array variables after invoking \java{Arrays.copyOf}.

\begin{figure}[!ht]
\begin{center}
\includegraphics{figs/array4.pdf}
\caption{Memory diagram of two variables referring to different arrays.}
\label{fig.array4}
\end{center}
\end{figure}

Of course, we can replace our recursive method altogether by using \java{Arrays.copyOf} and \java{a.length} for the second argument.
The following line produces the same result shown in Figure~\ref{fig.array4}:

\begin{code}
double[] b = Arrays.copyOf(a, a.length);
\end{code}

The \java{Arrays} class provides many other useful methods like \java{Arrays.compare}, \java{Arrays.equals}, \java{Arrays.fill}, and \java{Arrays.sort}.
Take a moment to read the documentation by searching the web for \java{java.util.Arrays}.


\section{Traversing Arrays}
\label{traversal}

\index{traversal}

Many computations can be implemented performing an operation on each element.
Applying a method through the elements of an array is called a {\bf traversal}:

\begin{code}

public static square(int[] a, int index) {
  if (index == a.length) {
    return;
  }
  a *= a[index];
}

public static square(int[] a) {
  square(a, 0);
}

\end{code}

This example traverses an array and squares each element.
If we call \java{square(\{1,2,3,4,5\})} the array has the values \verb"{1, 4, 9, 16, 25}".

\index{search}

Another common pattern is a {\bf search}, which involves traversing an array and ``searching'' for a particular element.
For example, the following method takes an array and a value, and it returns the index where the value appears:

% TODO: comparing floating point values doesn't always work (credit: Fazl)

\begin{code}
public static int search(double[] array, double target, int index) {
    if (index == array.length) {
        return -1; // target was not found
    }
    if array[index] == target) {
        return index;
    }
    search(array, target, index + 1);
}

public static int search(double[] array, double target) {
    search(array, target, 0);
}
\end{code}

If we find the target value in the array, we return its index immediately and stop traversing
through the array.
If the target is not in the array, the method returns \java{-1}, a special value chosen to indicate a failed search.
(This code is essentially what the \java{String.indexOf} method does.)

The following code searches an array for the value \java{1.23}, which is the third element.
Because array indexes start at 0, the output is \java{2}:

\begin{code}
double[] array = {3.14, -55.0, 1.23, -0.8};
int index = search(array, 1.23);
System.out.println(index);
\end{code}

\index{reduce}

Another common traversal is a {\bf reduce} operation, which ``reduces'' an array of values down to a single value.
Examples include the sum or product of the elements, the minimum, and the maximum.
The following method takes an array and returns the sum of its elements:

\begin{code}
public static double sum(double[] array, int index) {
  if (index == array.length) {
    return 0.0;
  }
  return array[index] + sum(array, index+1)

}
public static double sum(double[] array) {
    return sum(array, 0);
}
\end{code}

\section{Random Numbers}
\label{random}

\index{deterministic}

Most computer programs do the same thing every time they run; programs like that are called {\bf deterministic}.
Usually, determinism is a good thing, since we expect the same calculation to yield the same result.
But for some applications, we want the computer to be unpredictable.
Games are an obvious example, but there are many others, like scientific simulations.

%Technically speaking, all computer programs are deterministic: they simply execute the source code.

\index{nondeterministic}
\index{pseudorandom}

Making a program {\bf nondeterministic} turns out to be hard, because it's impossible for a computer to generate truly random numbers.
But there are algorithms that generate unpredictable sequences called {\bf pseudorandom} numbers.
For most applications, they are as good as random.

%Nondeterminism is a theoretical concept for analyzing the complexity of algorithms.

\index{Random}
\index{nextInt!Random}

If you did Exercise~\ref{guess}, you have already seen \java{java.util.Random}, which generates pseudorandom numbers.
The method \java{nextInt} takes an integer argument, \java{n}, and returns a random integer between \java{0} and \java{n - 1} (inclusive).

If you generate a long series of random numbers, every value should appear, at least approximately, the same number of times.
One way to test this behavior of \java{nextInt} is to generate a large number of values, store them in an array, and count the number of times each value occurs.

The following method creates an \java{int} array and fills it with random numbers between 0 and 99.
The argument specifies the desired size of the array, and the return value is a reference to the new array:

\begin{code}
public static void randomArrayValue(int[] a, int index) {
    if (index == a.length) {
        return;
    }
    Random random = new Random();
    a[index] = random.nextInt(100);
    return randomArrayValue(a, index + 1);
    
}

public static int[] randomArray(int size) {
    int[] a = new int[size];
    return randomArrayValue(a, 0);
}
\end{code}

The following \java{main} method generates an array and displays it by using the \java{printArray} method from Section~\ref{printarray}.
We could have used \java{Arrays.toString}, but we like seeing curly braces instead of square brackets:

\begin{code}
public static void main(String[] args) {
    int[] array = randomArray(8);
    printArray(array);
}
\end{code}

Each time you run the program, you should get different values.
The output will look something like this:

\begin{stdout}
{15, 62, 46, 74, 67, 52, 51, 10}
\end{stdout}

\section{Counting Characters}

We now return to the example from the beginning of the chapter and present a solution to Exercise~\ref{doubloon} using arrays.
Here is the problem again:

\begin{quote}
A word is said to be a ``doubloon'' if every letter that appears in the word appears exactly twice.
Write a method called \java{isDoubloon} that takes a string and checks whether it is a doubloon.
To ignore case, invoke the \java{toLowerCase} method before checking.
\end{quote}

Based on the approach from Section~\ref{singlepass}, we will create an array of 26 integers to count how many times each letter appears.
We convert the string to lowercase, so that we can treat \java{'A'} and \java{'a'} (for example) as the same letter.

\begin{code}
int[] counts = new int[26];
String lower = s.toLowerCase();
\end{code}

To update the \java{counts} array, we need to compute the index that corresponds to each character.
Fortunately, Java allows you to perform arithmetic on characters:

\begin{code}
public static boolean isDoubloon(int[] counts, String str, int index) {
    if (index == str.length) {
        return true;
    }
    char letter = lower.charAt(index);
    int letterInt = letter - 'a';
    counts[index]++;
    if (counts[index] > 2) {
      return false;
    }
    return isDouble(counts, str, index+1);
}
\end{code}

If \java{letter} is \java{'a'}, the value of \java{index} is \java{0};
if \java{letter} is \java{'b'}, the value of \java{index} is \java{1},
and so on.

Then we use \java{index} to increment the corresponding element of \java{counts}.
At the end of this recursive method, \java{counts} contains a histogram of the letters in the string \java{lower}.

Once we have the counts, we can use a second recursive method to check whether each letter appears zero or two times:

\begin{code}
public static boolean contains0or2(int[] counts, int index) {
  if (index == counts.length) {
    return true; // is a doubloon
  }
  if (counts[index] != 0 && counts[index] != 2) {
    return false; // is not a doubloon
  }
  return contains0or2(counts, index+1);

}
\end{code}

If we find a count that is neither 0 or 2, we know the word is not a doubloon and we can return immediately.
If we make it all the way through the last recursive call, we know that all counts are 0 or 2, which means the word is a doubloon.

Pulling together the code fragments, and adding some comments and test cases, here's the entire program:

\index{Doubloon.java}

\begin{trinket}{Doubloon.java}
public class Doubloon {

    public static boolean isDoubloon(int[] counts, String str, int index) {
        if (index == str.length) {
            return true;
        }
        char letter = lower.charAt(index);
        int letterInt = letter - 'a';
        counts[index]++;
        if (counts[index] > 2) {
          return false;
        }
        return isDouble(counts, str, index+1);
    }

    public static boolean contains0or2(int[] counts, int index) {
        if (index == counts.length) {
            return true; // is a doubloon
        }
        if (counts[index] != 0 && counts[index] != 2) {
            return false; // is not a doubloon
        }
        return contains0or2(counts, index+1);

    }

    public static boolean isDoubloon(String s) {
        return isDoubloon(s, 0);
    }

    public static void main(String[] args) {
        System.out.println(isDoubloon("Mama"));  // true
        System.out.println(isDoubloon("Lama"));  // false
    }
}
\end{trinket}

% ABD: Nice test cases :)

This example uses methods, \java{if} statements, recursive methods, arithmetic and logical operators, integers, characters, strings, booleans, and arrays.
We hope you'll take a second to appreciate how much you've learned!

\section{Count of value}
Let's look at this problem from CodingBat, available at \url{https://codingbat.com/prob/p135988}

\begin{quote}
\textbf{Recursion-1 ~array11}

Given an array of ints, compute recursively the number of times that the value 11 appears in the array.

\ttfamily
array11([1, 2, 11], 0) $\rightarrow$ 1 \\
array11([11, 11], 0) $\rightarrow$ 2 \\
array11([1, 2, 3, 4], 0) $\rightarrow$ 0
\end{quote}

This problem uses the convention of passing the index as an argument.
So the base case is when we've reached the end of the array.
At that point, we know there are no more {\tt 11}s:

\begin{code}
if (index >= nums.length) {
    return 0;
}
\end{code}

Next we look at the current number (based on the given index), and check if it's an {\tt 11}.
After that, we can recursively check the rest of the array.
Similar to the \java{noX} problem, we look at only one integer per method call:

\begin{code}
int recurse = array11(nums, index + 1);
if (nums[index] == 11) {
    return recurse + 1;
} else {
    return recurse;
}
\end{code}

Again, you can run this solution on CodingBat by pasting the snippets into the method definition.

\section{Vocabulary}
\todo{update these}

\begin{description}
\term{array}
A collection of values in which all the values have the same type, and each value is identified by an index.

\term{element}
One of the values in an array.
The \java{[]} operator selects elements.

\term{index}
An integer variable or value used to indicate an element of an array.

\term{allocate}
To reserve memory for an array or other object.
In Java, the \java{new} operator allocates memory.

\term{reference}
A value that indicates a storage location.
In a memory diagram, a reference appears as an arrow.

\term{alias}
A variable that refers to the same object as another variable.

\term{traversal}
Looping through the elements of an array (or other collection).

\term{search}
A traversal pattern used to find a particular element of an array.

\term{reduce}
A traversal pattern that combines the elements of an array into a single value.

\term{accumulator}
A variable used to accumulate results during a traversal.

\term{deterministic}
A program that does the same thing every time it is run.

\term{nondeterministic}
A program that always behaves differently, even when run multiple times with the same input.

\term{pseudorandom}
A sequence of numbers that appear to be random but are actually the product of a deterministic computation.

\term{histogram}
An array of integers in which each integer counts the number of values that fall into a certain range.

\term{enhanced for loop}
An alternative syntax for traversing the elements of an array (or other collection).

\end{description}


\section{Exercises}

The code for this chapter is in the {\it ch07} directory of {\it ThinkJavaCode2}.
See page~\pageref{code} for instructions on how to download the repository.
Before you start the exercises, we recommend that you compile and run the examples.

If you haven't already, take a look at Appendix~\ref{debugging}, where we've collected some of our favorite debugging advice.
It refers to language features we haven't yet covered, but it's good for you to know what's available when you need it.


\begin{exercise}  %%V6 Ex8.2

The purpose of this exercise is to practice reading code and recognizing the traversal patterns in this chapter.
The following methods are hard to read, because instead of using meaningful names for the variables and methods, they use names of fruit.

For each method, write one sentence that describes what the method does, without getting into the details of how it works.
And for each variable, identify the role it plays.

\begin{code}
public static int banana(int[] a) {
    int kiwi = 1;
    int i = 0;
    while (i < a.length) {
        kiwi = kiwi * a[i];
        i++;
    }
    return kiwi;
}
\end{code}

\begin{code}
public static int grapefruit(int[] a, int grape) {
    for (int i = 0; i < a.length; i++) {
        if (a[i] == grape) {
            return i;
        }
    }
    return -1;
}
\end{code}

\begin{code}
public static int pineapple(int[] a, int apple) {
    int pear = 0;
    for (int pine: a) {
        if (pine == apple) {
            pear++;
        }
    }
    return pear;
}
\end{code}

\end{exercise}


\begin{exercise}  %%V6 Ex8.3

What is the output of the following program?
Describe in a few words what \java{mus} does.
Draw a stack diagram just before \java{mus} returns.
%that shows the state of the program

\begin{code}
public static int[] make(int n) {
    int[] a = new int[n];
    for (int i = 0; i < n; i++) {
        a[i] = i + 1;
    }
    return a;
}
\end{code}

\begin{code}
public static void dub(int[] jub) {
    for (int i = 0; i < jub.length; i++) {
        jub[i] *= 2;
    }
}
\end{code}

\begin{code}
public static int mus(int[] zoo) {
    int fus = 0;
    for (int i = 0; i < zoo.length; i++) {
        fus += zoo[i];
    }
    return fus;
}
\end{code}

\begin{code}
public static void main(String[] args) {
    int[] bob = make(5);
    dub(bob);
    System.out.println(mus(bob));
}
\end{code}

\end{exercise}


\begin{exercise}  %%V6 Ex8.4

Write a method called \java{indexOfMax} that takes an array of integers and returns the index of the largest element.
Can you write this method by using an enhanced \java{for} loop?
Why or why not?

\end{exercise}


\begin{exercise}  %%V6 Ex8.5

The Sieve of Eratosthenes is ``a simple, ancient algorithm for finding all prime numbers up to any given limit'' (\url{https://en.wikipedia.org/wiki/Sieve_of_Eratosthenes}).

Write a method called \java{sieve} that takes an integer parameter, \java{n}, and returns a \java{boolean} array that indicates, for each number from \java{0} to \java{n - 1}, whether the number is prime.

\end{exercise}


\begin{exercise}  %%V6 Ex8.6

Write a method named \java{areFactors} that takes an integer \java{n} and an array of integers, and returns \java{true} if the numbers in the array are all factors of \java{n} (which is to say that \java{n} is divisible by all of them).

\end{exercise}


\begin{exercise}  %%V6 Ex8.7

Write a method named \java{arePrimeFactors} that takes an integer \java{n} and an array of integers, and that returns \java{true} if the numbers in the array are all prime {\it and} their product is \java{n}.

\end{exercise}


\begin{exercise}  %%V6 Ex9.2

Write a method called \java{letterHist} that takes a string as a parameter and returns a histogram of the letters in the string.
The zeroth element of the histogram should contain the number of a's in the string (upper- and lowercase); the 25th element should contain the number of z's.
Your solution should traverse the string only once.

\end{exercise}

\begin{exercise}  %%V6 Ex8.8

Many of the patterns you have seen for traversing arrays can also be written recursively.
It is not common, but it is a useful exercise.

\begin{enumerate}

\item Write a method called \java{maxInRange} that takes an array of integers and two indexes, \java{lowIndex} and \java{highIndex}, and finds the maximum value in the array, but considering only the elements between \java{lowIndex} and \java{highIndex}, including both.

This method should be recursive.
If the length of the range is 1 (i.e., if \java{lowIndex == highIndex}), we know immediately that the sole element in the range must be the maximum.
So that's the base case.

If there is more than one element in the range, we can break the array into two pieces, find the maximum in each piece, and then find the maximum of the maxima.

\item Methods like \java{maxInRange} can be awkward to use.
To find the largest element in an array, we have to provide the range for the entire array:

\begin{code}
double max = maxInRange(a, 0, a.length - 1);
\end{code}

Write a method called \java{max} that takes an array and uses \java{maxInRange} to find and return the largest element.

\end{enumerate}

\end{exercise}

\begin{exercise}  %%V6 Ex9.7

\index{anagram}

Two words are anagrams if they contain the same letters and the same number of each letter.
For example, ``stop'' is an anagram of ``pots'', ``allen downey'' is an anagram of ``well annoyed'', and ``christopher mayfield'' is an anagram of ``hi prof the camel is dry''.
Write a method that takes two strings and checks whether they are anagrams of each other.

\end{exercise}
